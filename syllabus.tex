\documentclass[11pt]{article}
%\usepackage{geometry}
\usepackage[inner=1.5cm,outer=1.5cm,top=2.5cm,bottom=2.5cm]{geometry}
\pagestyle{empty}
\usepackage{graphicx}
\usepackage{fancyhdr, lastpage, bbding, pmboxdraw}
\usepackage[usenames,dvipsnames]{color}
\usepackage{longtable}
\usepackage{multicol}
\usepackage{multirow}
\usepackage{lscape}
\usepackage{booktabs}
\definecolor{darkblue}{rgb}{0,0,.6}
\definecolor{darkred}{rgb}{.7,0,0}
\definecolor{darkgreen}{rgb}{0,.6,0}
\definecolor{red}{rgb}{.98,0,0}
\usepackage[colorlinks,pagebackref,pdfusetitle,urlcolor=darkblue,citecolor=darkblue,linkcolor=darkred,bookmarksnumbered,plainpages=false]{hyperref}
\renewcommand{\thefootnote}{\fnsymbol{footnote}}

\pagestyle{fancyplain}
\fancyhf{}
\lhead{ \fancyplain{}{MEEN 315 Principles of Thermodynamics, Spr 2017} }
%\chead{ \fancyplain{}{} }
\rhead{ \fancyplain{}{\today} }
%\rfoot{\fancyplain{}{page \thepage\ of \pageref{LastPage}}}
\fancyfoot[RO, LE] {page \thepage\ of \pageref{LastPage} }
\thispagestyle{plain}

%%%%%%%%%%%% LISTING %%%
\usepackage{listings}
\usepackage{caption}
\DeclareCaptionFont{white}{\color{white}}
\DeclareCaptionFormat{listing}{\colorbox{gray}{\parbox{\textwidth}{#1#2#3}}}
\captionsetup[lstlisting]{format=listing,labelfont=white,textfont=white}
\usepackage{verbatim} % used to display code
\usepackage{fancyvrb}
\usepackage{acronym}
\usepackage{amsthm}
\VerbatimFootnotes % Required, otherwise verbatim does not work in footnotes!


%%%%%%%%%%%%%%%%%%%%%%%%%%%%%%%%%%%%
\begin{document}
\begin{center}
{\Large \textsc{MEEN 315, Principles of Thermodynamics, sec. 505}}
\end{center}
\begin{center}
Spring 2017
\end{center}
%\date{September 26, 2014}

\begin{center}
\rule{6in}{0.4pt}
\begin{minipage}[t]{.75\textwidth}
\begin{tabular}{llcccll}
\textbf{Instructor:} & Mitchell Paulus & & &  & \textbf{Time:} & MWF 4:10-5:00 PM \\
\textbf{Email:} &  \href{mailto:paulusm14@gmail.com}{paulusm14@gmail.com} & & & & \textbf{Place:} & RICH 114  \\
\textbf{GitHub username:} & @mitchpaulus &&&  & \textbf{Office:} & ESL RM 187 \\
\end{tabular}
\end{minipage}
\rule{6in}{0.4pt}
\end{center}
\vspace{.5cm}
\setlength{\unitlength}{1in}
\renewcommand{\arraystretch}{1}

\noindent \textbf{Required or Elective:} Required course. 


\vspace{0.15in}
\noindent\textbf{Objectives/Catalog Description:} (3-0) Credit 3. 
Theory and application of energy methods in engineering, energy transfer by heat, work and mass;
thermodynamic properties;
analysis of open and closed systems;
the second law of thermodynamics and entropy;
gas, vapor and refrigeration cycles; and applications.


\vspace{0.15in}
\noindent\textbf{Class/Laboratory Schedule:}
Three 50 minute
sessions per week, taught in lecture style. Project and problem-solving
teaming emphasized. Teams of three or four are organized for
out-of-class exercises.


%%\noindent\textbf{Course Pages:} \begin{enumerate}
%%\item \url{youtube.com/c/MitchellPaulus}
%%\end{enumerate}

\vskip.15in
\noindent\textbf{Office Hours:} Mon. 5-6, Tues. 11-12, Wed. 3-4, or by appointment.

\vskip.15in
\noindent\textbf{Main References:} %\footnotemark
This is a  restricted list of various interesting and useful books that
will be touched during the course. You need to consult them
occasionally.
\begin{itemize}
    \item Moran, Shapiro, Boettner and Bailey. 2014. \textit{Fundamentals of Engineering Thermodynamics, 8th edition}.
\end{itemize} 

% \footnotetext{Downloadable ebook versions are available on AeLP.}


\vskip.15in
\noindent\textbf{Prerequisites:}
MEEN 221 or MEEN 225 (Engineering Mechanics), and MATH 251 or 253 (Engineering Mathematics III).
It is the student's responsibility to ensure proper requirements are satisfied for enrollment in this course.
Students not meeting course pre-requisites will be automatically dropped after the first week of class.

\vspace*{0.15in}
\noindent \textbf{Course Learning Outcomes:} At the end of the course, students should be able to:

\begin{enumerate}
    \item look up thermodynamic properties in tables;
    \item construct pressure-temperature, pressure-volume, or temperature-volume phase diagrams for pure substances;
    \item indicate a process on an appropriate phase diagram;
    \item use compressibility charts;
    \item calculate expansion/compression work in a closed system;
    \item apply conservation of energy to a closed system to determine heat transfer, work, or property changes;
    \item use conservation of mass to determine change in mass of a system;
    \item apply conservation of energy to an open system to determine heat transfer, work, or property changes;
    \item analyze first law performance of simple engineering devices (valves, turbines, boilers, etc.); 
    \item determine maximum performance of cycles using the Carnot cycle;
    \item identify sources of entropy generation in a system;
    \item calculate work for isentropic processes;
    \item calculate isentropic efficiencies of simple engineering devices: turbines, compressors, pumps, etc.
    \item identify work/heat processes in any arbitrary cycle;
    \item estimate work and efficiency for the Otto, Diesel, and Brayton cycles;
    \item estimate work and efficiency for the Rankine power cycle; 
    \item team with other students to solve thermodynamic problems and write a technical paper.
\end{enumerate}



\vspace*{.15in}
\noindent\textbf{Examinations:}
Two midterm exams and a comprehensive final exam are scheduled.
Unexcused absences will result in a grade of zero for missed examinations.
Known absences for a scheduled exam must be brought to the attention of the instructor as soon as possible.
The second exam is comprehensive with emphasis on material covered since the first exam.
The final exam is comprehensive with emphasis on the material covered since the second exam.
YOU WILL NOT RECEIVE YOUR FINAL EXAM, OR A COPY OF YOUR FINAL EXAM, AT THE END OF THE SEMESTER.
You may stop by my office to review your graded final exam after establishing an appointment with me.


\vspace*{.15in}
\noindent\textbf{Grading Policy:}

\begin{tabular}{rl}
    Homework  & 15\%   \\
    Quizzes  & 10\%    \\
    Attendance & 5\% \\
    Midterm 1 & 20\% \\
    Midterm 2 &  20\% \\ 
    Final & 30\% \\ %Four Projects (40\% = 4 * 10\%)
\end{tabular}



\vskip.15in
\noindent\textbf{Important Dates:}
\begin{center} \begin{minipage}{3.8in}
\begin{flushleft}
Midterm \#1      \dotfill Feb 22, 7:00-9:00 PM \\
Midterm \#2      \dotfill April 5, 7:00-9:00 PM \\
%Project Deadline \dotfill ~Month Day \\
Final Exam       \dotfill  TBA  \\
\end{flushleft}
\end{minipage}
\end{center}

\vskip.15in \noindent\textbf{Attendance:}  Attendance is essential and
expected. Short written quizzes will be given the beginning of each
class in order to track attendance. A student can miss up to 2 classes
and still receive full credit for attendance.


\vskip.15in
\noindent\textbf{Honor Code:}
``An Aggie does not lie, cheat, or steal, or tolerate those who do.''
Upon accepting admission to Texas A\&M University, a student immediately assumes a commitment to uphold the Honor Code, to accept responsibility for learning and to follow the philosophy and rules of the Honor System.
Students will be required to state their commitment on examinations, research papers, and other academic work.
Ignorance of the rules does not exclude any member of the Texas A\&M University community from the requirements or the processes of the Honor System.
For additional information please visit: aggiehonor.tamu.edu

On all course work, assignments, and examinations at Texas A\&M University, the following Honor Pledge is implied regardless if it is preprinted and signed by the student: 
``On my honor, as an Aggie, I have neither given nor received unauthorized aid on this academic work.''


\vskip.15in
\noindent\textbf{ADA:}
The Americans with Disabilities Act (ADA) is a federal antidiscrimination statute that provides comprehensive civil rights protection for persons with disabilities.
Among other things, this legislation requires that all students with disabilities be guaranteed a learning environment that provides for reasonable accommodation of their disabilities.
If you believe you have a disability requiring an accommodation, please contact Disability Services, currently located in the Disability Services building at the Student Services at White Creek complex on west campus or call 9798451637.
For additional information, visit http://disability.tamu.edu.

\vspace{0.15in}
\noindent \textbf{Contribution to Meeting Requirement of Criterion 5:}

\begin{tabular}{cc|cc|cc}
Subject       & Semester hrs & Subject             & Semester hrs & Subject & Semester hrs  \\ \hline
Mathematics   &              & Engineering Science & 3            & General & \\ \hline
Basic Science &              & Engineering Design  &              &         & \\ \hline
\end{tabular}

\vspace{0.15in}
\noindent \textbf{Relationship of Course to Program Outcomes} 
\begin{enumerate}
    \item \textbf{ability to apply knowledge of mathematics, science and engineering}
    \item ability to design and construct experiments, as well as to analyze and interpret data
    \item ability to design a system, component, or process to meet desired
        needs within realistic constraints such as economic, environmental,
        social, political, ethical, health and safety, manufacturability, and
        sustainability
    \item \textbf{ability to function on multi-disciplinary teams}
    \item \textbf{ability to identify, formulate and solve engineering problems}
    \item understanding of professional and ethical responsibility
    \item ability to communicate effectively
    \item broad education necessary to understand the impact of engineering
        solutions in a global, economic, environmental, and societal context
    \item recognition of the need for, and an ability to engage in life-long learning
    \item knowledge of contemporary issues
    \item ability to use the techniques, skills, and modern engineering tools necessary for engineering practice
\end{enumerate}



\clearpage

\begin{landscape}
\noindent\textbf{Tentative Schedule}
{
    \small
    \begin{longtable}{llllp{8cm}lp{7cm}}
\multicolumn{2}{c}{Lecture/Week} &   & Date   & Topic (Lecture Coverage)                                                       & Text Coverage & Notes                                                                   \\ \midrule \endhead
1    & 1  & W & 18-Jan & Introduction, Concepts and Definitions                                         &       & HW 1 Out                                                                \\
2    & 1  & F & 20-Jan & Units, Dimensions, Volume, Pressure                                            &       & \\
3    & 2  & M & 23-Jan & Temperature, Problem Solving Methodology                                       &       & \\
4    & 2  & W & 25-Jan & Mechanical Concepts of Energy, Basic Work Processes                            &       & HW 1 DUE; HW 2 Out                                                      \\
5    & 2  & F & 27-Jan & Energy Transfer by Heat, Closed System Energy Balance                          &       & \\
6    & 3  & M & 30-Jan & Cycle Energy Analysis                                                          &       & \\
7    & 3  & W & 1-Feb  & Phases, Fixing the State                                                       &       & HW 2 DUE; HW 3 Out                                                      \\
9    & 3  & F & 3-Feb  & Using Tables for Pressure, Temperature, and Specific Volume                    &       & \\
10   & 4  & M & 6-Feb  & Using Tables of Energy and Enthalpy, Energy Balance with Properties            &       & HW 3 DUE; HW 4 Out.                                                     \\
11   & 4  & W & 8-Feb  & Specific Heats, Evaluating Properties of Solids and Liquids                    &       & \\
12   & 4  & F & 10-Feb & General Compressibility Charts and Ideal Gas Model                             &       & \\
13   & 5  & M & 13-Feb & Property Changes and Energy Balances of Ideal Gases                            &       & HW 4 DUE, HW 5 Out (HW5 material is covered on Exam 1)                  \\
14   & 5  & W & 15-Feb & Polytropic Processes                                                           &       & End of Exam 1 Material (through Lecture 14). Sample exam posted online. \\
15   & 5  & F & 17-Feb & Conservation of Mass                                                           &       & \\
16   & 6  & M & 20-Feb & Control Volume Conservation of Energy and Steady-State Analysis                &       & HW 5 DUE.                                                               \\
     &    & W & 22-Feb & Optional Review in Class                                                       &       & \\
17   & 6  & W & 22-Feb & \textbf{EXAM 1}                                                                &       & EXAM 1 (7 - 9 PM), Location TBA                                         \\
18   & 6  & F & 24-Feb & Nozzles and Diffusers                                                          &       & \\
19   & 7  & M & 27-Feb & Turbines, Compressors, and Pumps                                               &       & HW 6 Out.                                                               \\
20   & 7  & W & 1-Mar  & Heat Exchangers and Throttling Devices                                         &       & \\
21   & 7  & F & 3-Mar  & System Integration and Transient Analysis                                      &       & \\
22   & 8  & M & 6-Mar  & Introducing the Second Law, Statements of the Second Law                       &       & HW 6 DUE; HW 7 Out                                                      \\
23   & 8  & W & 8-Mar  & Irreversible and Reversible Processes, Interpreting the Kelvin Plank Statement &       & \\
24   & 8  & F & 10-Mar & Second Law applied to Thermodynamic Cycles, Heat Engines                       &       & \\
     &    &   &        & \textbf{SPRING BREAK MAR. 13-17}                                               &       & \\
25   & 9  & M & 20-Mar & Refrigeration / Heat Pumps, Maximum Performance                                &       & HW 7 DUE; HW 8 Out                                                      \\
26   & 9  & W & 22-Mar & Carnot Cycle and Clausius Inequality                                           &       & \\
27   & 9  & F & 24-Mar & Entropy and Property Data                                                      &       & \\
28   & 10 & M & 27-Mar & TdS Equations, Entropy Changes for Solids and Liquids                          &       & HW 8 DUE; HW 9 Out (HW 9 material is covered on Exam 2).                \\
29   & 10 & W & 29-Mar & Entropy Changes of Ideal Gases, Internally Reversible Closed Processes         &       & End of Exam 2 material (through Lecture 29). Sample exam posted online. \\
30   & 10 & F & 31-Mar & Closed System Entropy Balance                                                  &       & \\
31   & 11 & M & 3-Apr  & Increase in Entropy Principle                                                  &       & HW9 DUE                                                                 \\
     &    & W & 5-Apr  & Optional Review in Class                                                       &       & \\
32   & 11 & W & 5-Apr  & \textbf{EXAM 2}                                                                &       & EXAM 2 (7 - 9 PM), Location TBA                                         \\
33   & 11 & F & 7-Apr  & Control Volume Entropy Equation, Steady State Analysis                         &       & \\
34   & 12 & M & 10-Apr & Isentropic Processes, Isentropic Efficiencies                                  &       & HW 10 Out.                                                              \\
35   & 12 & W & 12-Apr & Rankine Vapor Power Cycle                                                      &       & \\
     &    & F & 14-Apr & \textbf{Reading day, Good Friday, no class.}                                   &       & \\
36   & 13 & M & 17-Apr & Rankine Vapor Power Cycle                                                      &       & \\
37   & 13 & W & 19-Apr & Gas Power Cycles                                                               &       & HW 10 DUE; HW 11 Out.                                                   \\
38   & 13 & F & 21-Apr & Gas Power Cycles                                                               &       & \\
39   & 14 & M & 24-Apr & Refrigeration and Heat Pump Cycles                                             &       & \\
40   & 14 & W & 26-Apr & Refrigeration and Heat Pump Cycles                                             &       & \\
41   & 14 & F & 28-Apr & Cushion                                                                        & Notes & \\
42   & 15 & M & 1-May  & Conclusion, Course Evaluation                                                  & Notes & HW 11 DUE.                                                              \\
43   & 15 & T & 2-May  & Redined Day                                                                    &       & \\
     &    &   &        & Final Exam Review                                                              &       & TBD                                                                     \\
     &    &   &        & \textbf{FINAL EXAM}                                                                     &       & TIME (Room: Ordinary Classroom).
\end{longtable}
}
\end{landscape}

%%2&1&8/31
%%(W)&&1.4 � 1.6&&Video: Example 1 � English Units
%%Video: Lecture 1 � What is Thermodynamics?
%%3&1&9/2
%%(F)&Temperature, Problem Solving Methodology&1.7 � 1.9&&
%%4&2&9/5
%%(M)&Mechanical Concepts of Energy, Basic Work Processes&2.1 � 2.2&HW 1 DUE; HW 2 Out
%%Team Selections DUE&
%%5&2&9/7
%%(W)&Energy Transfer by Heat, Closed System Energy Balance&2.3 � 2.5&&Video: Example 2 - Electric Motors
%%Video: Example 3 - Wind Mills
%%6&2&9/9
%%(F)&Cycle Energy Analysis&2.6&&
%%7&3&9/12
%%(M)&Phases, Fixing the State&3.1&HW 2 DUE; HW 3 Out&
%%8&3&9/14
%%(W)&P-V-T Relationships, Phase Changes&3.2 � 3.3&&Video: Lecture 2 � Phase Change and Property Tables
%%9&3&9/16
%%(F)&Using Tables for Pressure, Temperature, and Specific Volume&3.4 � 3.5&
%%&Video: Example 4 - Using Tables (Water and R-134a)
%%10&4&9/19
%%(M)&Using Tables of Energy and Enthalpy, Energy Balance with Properties&3.6 � 3.8&HW 3 DUE; HW 4 Out.&
%%11&4&9/21
%%(W)&Specific Heats, Evaluating Properties of Solids and Liquids&3.9 � 3.10&&
%%12&4&9/23
%%(F)&General Compressibility Charts and Ideal Gas Model&3.11 � 3.12&&Video: Lecture 3 � Ideal Gases
%%13&5&9/26
%%(M)&Property Changes and Energy Balances of Ideal Gases&3.13 � 3.14&HW 4 DUE, HW 5 Out (HW5 material is covered on Exam 1)&
%%14&5&9/28
%%(W)&Polytropic Processes&3.15&End of Exam 1 Material (through Lecture 14). Sample exam posted online.&Video: Lecture 4 � Boundary Work
%%Video: Example 5 - Closed System Unknown State 2
%%15&5&9/30
%%(F)&Conservation of Mass
%%&4.1 � 4.3&&
%%16&6&10/3
%%(M)&Control Volume Conservation of Energy and Steady-State Analysis&4.4 � 4.5&HW 5 DUE.&
%%&&10/5
%%(W)&Optional Review in Class&&&
%%17&6&10/6
%%(R)&EXAM 1&&EXAM 1 (7 � 9 PM), Location TBA&
%%18&6&10/7
%%(F)&Nozzles and Diffusers&4.6&&Video: Lecture 6 � Nonwork SS Devices
%%Video: Example 7 - SS Non Work Device
%%19&7&10/10
%%(M)&Turbines, Compressors, and Pumps&4.7 � 4.8&HW 6 Out.
%%&Video: Lecture 5 � SS Work Devices
%%Video: Example 6 - SS Work Device
%%20&7&10/12
%%(W)&Heat Exchangers and Throttling Devices&4.9 � 4.10&&
%%21&7&10/14
%%(F)&System Integration and Transient Analysis&4.11 � 4.12&&
%%22&8&10/17
%%(M)&Introducing the Second Law, Statements of the Second Law&5.1 � 5.2&HW 6 DUE; HW 7 Out
%%&Video: Lecture 7 � 2nd Law Principles
%%23&8&10/19
%%(W)&Irreversible and Reversible Processes, Interpreting the Kelvin Plank Statement&5.3 � 5.4&&Video: Lecture 8 - Reversible Irreversible Processes
%%24&8&10/21
%%(F)&Second Law applied to Thermodynamic Cycles, Heat Engines&5.5 � 5.6&&
%%25&9&10/24
%%(M)&Refrigeration / Heat Pumps, Maximum Performance&5.7, 5.9&HW 7 DUE; HW 8 Out&
%%26&9&10/26
%%(W)&Carnot Cycle and Clausius Inequality&5.10 � 5.11&&
%%27&9&10/28
%%(F)&Entropy and Property Data&6.1 � 6.2&&
%%28&10&10/31
%%(M)&TdS Equations, Entropy Changes for Solids and Liquids&6.3 � 6.4&HW 8 DUE; HW 9 Out (HW 9 material is covered on Exam 2). 
%%&
%%29&10&11/2
%%(W)&Entropy Changes of Ideal Gases, Internally Reversible Closed Processes&6.5 � 6.6&End of Exam 2 material (through Lecture 29). Sample exam posted online.&
%%30&10&11/4
%%(F)&Closed System Entropy Balance&6.7&&
%%31&11&11/7
%%(M)&Increase in Entropy Principle&6.8&HW9 DUE&
%%&&11/9
%%(W)&Optional Review in Class&&&
%%32&11&11/10
%%(R)&EXAM 2&&EXAM 2 (7 � 9 PM), Location TBA&
%%33&11&11/11
%%(F)&Control Volume Entropy Equation, Steady State Analysis&6.9 � 6.10&&
%%34&12&11/14
%%(M)&Isentropic Processes, Isentropic Efficiencies&6.11 � 6.12&HW 10 Out.&Video: Lecture 9 � Isentropic Efficiency
%%Video: Example 8 - Isentropic Efficiency
%%Video: Example 9 � SS Device Polytropic Process
%%35&12&11/16
%%(W)&Rankine Vapor Power Cycle&8.1 � 8.2&&Video: Lecture 10 � Vapor Power Cycles
%%Video: Lecture 11 � Reheat and Air Cycles
%%36&12&11/18
%%(F)&Rankine Vapor Power Cycle&8.3 � 8.4&&
%%37&13&11/21
%%(M)&Gas Power Cycles&9.1 � 9.2&HW 10 DUE; HW 11 Out.&
%%&&11/23
%%(W)&Reading day, no class.&&&
%%&&11/24
%%(R)&Thanksgiving Holiday&&&
%%&&11/25
%%(F)&Thanksgiving Holiday, no class.&&&
%%38&14&11/28
%%(M)&Gas Power Cycles&9.3 � 9.4&&
%%39&14&11/30
%%(W)&Refrigeration and Heat Pump Cycles
%%&10.1 � 10.2
%%&&
%%40&14&12/2
%%(F)&Refrigeration and Heat Pump Cycles&10.3 � 10.4&&
%%41&13&12/5
%%(RF)&Cushion&Notes&&
%%42&13&12/7
%%(W)&Conclusion, Course Evaluation&Notes&HW 11 DUE.&
%%&&??&Final Exam Review&&TBD&
%%&&??&FINAL EXAM&&TIME (Room: Ordinary Classroom).&
%%


%%%%%% THE END 
\end{document} 
